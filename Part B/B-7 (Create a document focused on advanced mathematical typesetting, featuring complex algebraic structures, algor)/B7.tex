\documentclass[11pt]{article}

\usepackage[utf8]{inputenc}
\usepackage[T1]{fontenc}
\usepackage{amsmath, amssymb, amsfonts, amsthm} % For math fonts, symbols and environments
\usepackage{algorithm, algpseudocode} % For algorithms

% Theorem Styles
\theoremstyle{definition}
\newtheorem{definition}{Definition}[section]
\theoremstyle{remark}
\newtheorem*{remark}{Remark}
\theoremstyle{plain}
\newtheorem{theorem}{Theorem}[section]

% Custom operators
\DeclareMathOperator{\Rank}{Rank}
\DeclareMathOperator{\Span}{Span}

\title{Advanced Mathematical Typesetting in LaTeX}
\author{Your Name}
\date{\today}

\begin{document}

\maketitle

\section{Introduction}
Introduction to advanced mathematical typesetting.

\section{Complex Algebraic Structures}
Examples of complex algebraic structures:

\subsection{Matrices and Vectors}
\begin{equation}
    \mathbf{A} = 
    \begin{pmatrix}
        a_{11} & a_{12} \\
        a_{21} & a_{22}
    \end{pmatrix}, \quad
    \mathbf{v} = 
    \begin{pmatrix}
        v_1 \\
        v_2
    \end{pmatrix}
\end{equation}

\subsection{Custom Operators}
\begin{equation}
    \Rank(\mathbf{A}), \quad \Span(\mathbf{v})
\end{equation}

\section{Theorems and Definitions}
\begin{definition}[Euclidean Space]
    A Euclidean space is a finite-dimensional inner product space.
\end{definition}

\begin{theorem}
    Every finite-dimensional inner product space is a Euclidean space.
\end{theorem}

\begin{proof}
    The proof is left as an exercise for the reader.
\end{proof}

\begin{remark}
    This is an important theorem in linear algebra.
\end{remark}

\section{Algorithms}
\begin{algorithm}
\caption{Example Algorithm}
\begin{algorithmic}[1]
\Procedure{Example}{$a,b$}
    \State $r \gets a \bmod b$
    \While{$r \neq 0$}
        \State $a \gets b$
        \State $b \gets r$
        \State $r \gets a \bmod b$
    \EndWhile
    \State \textbf{return} $b$
\EndProcedure
\end{algorithmic}
\end{algorithm}

\end{document}
