% Document class setup
\documentclass[11pt,a4paper]{article}

% Packages for encoding and font
\usepackage[utf8]{inputenc}
\usepackage[T1]{fontenc}

% TikZ package for drawing diagrams
\usepackage{tikz}
% TikZ libraries for geometric shapes and arrow styles
\usetikzlibrary{shapes.geometric, arrows}

% Document metadata
\title{Document with a TikZ Flowchart}
\author{Your Name}
\date{\today}

% Beginning of the document
\begin{document}
	
	% Create the title
	\maketitle
	
	% Section for the flowchart
	\section{Flowchart Example}
	
	% Descriptive text
	Below is an example of a basic flowchart created using TikZ:
	
	% Define styles for flowchart elements
	\tikzstyle{startstop} = [rectangle, rounded corners, minimum width=3cm, minimum height=1cm, text centered, draw=black, fill=red!30]
	\tikzstyle{process} = [rectangle, minimum width=3cm, minimum height=1cm, text centered, draw=black, fill=orange!30]
	\tikzstyle{decision} = [diamond, minimum width=3cm, minimum height=1cm, text centered, draw=black, fill=green!30]
	\tikzstyle{arrow} = [thick,->,>=stealth]
	
	% Begin figure environment for the flowchart
	\begin{figure}[h]
		\centering % Center the figure in the text
		
		% Begin the TikZ picture environment
		\begin{tikzpicture}[node distance=2cm]
			
			% Nodes of the flowchart
			\node (start) [startstop] {Start};
			\node (pro1) [process, below of=start] {Process 1};
			\node (dec1) [decision, below of=pro1] {Decision 1};
			\node (pro2a) [process, below of=dec1, left of=dec1] {Process 2A};
			\node (pro2b) [process, below of=dec1, right of=dec1] {Process 2B};
			\node (stop) [startstop, below of=dec1, yshift=-2cm] {Stop};
			
			% Arrows between nodes
			\draw [arrow] (start) -- (pro1);
			\draw [arrow] (pro1) -- (dec1);
			\draw [arrow] (dec1) -| node[anchor=east] {yes} (pro2a);
			\draw [arrow] (dec1) -| node[anchor=west] {no} (pro2b);
			\draw [arrow] (pro2a) |- (stop);
			\draw [arrow] (pro2b) |- (stop);
			
			% End of the TikZ picture environment
		\end{tikzpicture}
		
		% Caption for the flowchart
		\caption{A simple TikZ flowchart}
		
		% End of the figure environment
	\end{figure}
	
	% End of the document
\end{document}
